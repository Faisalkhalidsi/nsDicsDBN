\documentclass[conference]{IEEEtran}
\renewcommand\IEEEkeywordsname{Keywords}
\usepackage{natbib}
\usepackage{amsmath}


\usepackage{caption}
\captionsetup{labelsep=space,justification=justified,singlelinecheck=off}
% correct bad hyphenation here
\hyphenation{op-tical net-works semi-conduc-tor}


\begin{document}

\title{Combination Normal Sparse and Discriminative Deep Belief Networks}

\author{\IEEEauthorblockN{Faisal Khalid}
\IEEEauthorblockA{Department of Computer Science\\
University of Indonesia\\
Depok, West Java, Indonesia\\
faisal.khalid.si@gmail.com}
\and

\IEEEauthorblockN{Mohamad Ivan Fanany}
\IEEEauthorblockA{Department of Computer Science\\
University of Indonesia\\
Depok, West Java, Indonesia\\
ivan@cs.ui.ac.id}}

% make the title area
\maketitle

% As a general rule, do not put math, special symbols or citations
% in the abstract
\begin{abstract}
Problem in deep models is time consuming and may
be trapped into local minima. One useful tool for dealing with his
problem to use DBN (Deep Belief Network). Sparse
representations are more efficient than non-sparse. This paper
aim to find the best structure of combination sparsity and
discriminative DBN. We use DBN architecture with 784 as input,
500 hidden unit, 500 hidden unit, 2000 hidden unit, and 10 as
output. We took 3 step in this experiment, preliminary,
intermediate and final result. Every analysis of each step be a
background to the next step. We use normal sparse for sparse
generative and sparse discriminative. Experimental studies on
MNIST dataset show that the best structure on combination
normal sparse and discriminative deep belief networks is with
input- generative (Contrastive Divergence) - generative
(Contractive Divergence) – normal sparse
discriminative
(Contractive Divergence).
\end{abstract}

\begin{IEEEkeywords}
Deep Belief Network, Normal Sparse Discriminative Deep Belief Network
\end{IEEEkeywords}
% no keywords

\IEEEpeerreviewmaketitle



\section{Introduction}
Artificial neural network have been used in pattern
recognition, voice recognition and natural language
processing. For some theoretical reason deep architecture were
suggested \cite{keyvanrad1}.
Problem in deep models is time consuming and may be
trapped into local minima. To solve that problem, we can use
DBN (Deep Belief Network). DBN can create a neural
networks which have several hidden layers \cite{liu1}.
Computing connection between hidden layer and visible
layer in DBN is so difficult. Gibbs sampling can be used,
although it takes a long time. So it is impossible to
do.Contrastive Divergence (CD)\cite{carreiraperpinan1}, PCD \cite{tieleman1}, and FEPCD \cite{keyvanrad2}
are used to solved that problem.
In recent years there has been an increasing interest in
automatic feature extraction. Many models were then
developed for such as deep learning with interest in sparse
coding\cite{olshausen1}  methods.
Sparsity has recently become a concept of great interest
and has become a key ingredient in Deep Belief Network.
Bengio has argued that sparse representations are more
efficient than non-sparse in fixed size representations \cite{bengio1}.
Several previous studies have done in sparse and
discriminative deep belief network. The studies in\cite{tieleman1}
proposed semi-supervise learning by using discriminative
Deep Belief Network(DDBN). \cite{walid1} proposed a sparse deep
belief networks and denoising autoencoder to a new dataset
proposed in the ICDAR 2013 handwritten digit
competition.\cite{larochelle1} using discriminative RBM as classifiers can
improve performance in a semi-supervised learning. The
studies in\cite{halkias1} present a theoretical approach for sparse
constraints in the DBN using the mixed norm for both-
overlapping and overlapping groups.
We argue that combination sparsity and discriminative
DBN will increase the accuracy, but no one ever proposed
what structure of that combination will give the best accuracy.
This paper aim to find the best structure of combination
sparsity and discriminative DBN. For ease of reproduction of
results, this paper are performed on publicly available dataset
(MNIST Dataset). This paper organized as follows. Section II
explain literature study. Section III describes proposed
methods. Section IV describe the experimental setup. Section
V explain the experimental result and discussions. Finally
conclusions are given in section VI.

\section{Literature Review}
\subsection{Normal Sparse RBM}
Essentially
RBMs
learn
non-sparse
distributed
representations. In all proposed methods, learning algorithm in
RBM has been changed to enforce RBM to learn sparse
representation. The goal of sparsity in RBM is that most of
hidden units have zero values and this equivalent to force
activation probability of hidden units to zero. Normal sparse
regulation term has a variance parameter that can control the
force degree of sparseness. The regulation term used normal
function properties\cite{keyvanrad1}.
In another paper based on rate distortion theory, the
penalty factor is activation probability in hidden unit\cite{ji1}.
Algorithm sparse RBM learning algorithm\cite{lee1}
\begin{enumerate}
	\item Update the parameters using approximation to the
	gradient of log likelihood like CD, PCD or FEPCD.
	\item Update the parameter using gradient of the regulation
	term.
	\item Repeat steps 1 and 2 until convergence on reach to
	max epoch.
\end{enumerate}
In \cite{keyvanrad1} proposed a new regulation term that has different
behavior according to deviation of the activation of the hidden
units from a (low) fixed level p. They used normal probability
density function as regulation term.
\begin{equation}\label{eq:1}
L_{sparsity}=\sum_{j=1}^{n}f(q_{j},p,\sigma^{2})
\end{equation}
According to \ref{eq:1} the gradient if regularization
term must be used in step 2. The gradient of regularization
term can be computed as follows :
\begin{multline}
\frac{\partial}{\partial b_{j}}L_{sparsity }\oe  \frac{1}{m}\left ( p-\frac{1}{m}\sum_{l=1}^{m}q_{j}^{(l)} \right )f\left ( q_{j},p,\sigma^{2}   \right )\\ \times\sum_{l=1}^{m}q_{j}^{(l)}\left ( 1-q_{j}^{(l)} \right )
\end{multline}

\subsection{Discriminative Restricted Boltzmann Machines (DRBM)}
The important thing in classification is have a good and
correct classification. we can optimize directly 
$ p\left ( x|y \right )$  below\cite{larochelle1}
\begin{equation}
L_{disc}\left ( D_{train} \right )= -\sum_{i=1}^{\left | D_{train} \right |}log    p\left ( y_{i}|x_{i} \right )
\end{equation}
Based on $L_{disc}$, discriminative RBM (DRBM) can refer to
RBMs trained. Contrastive divergence can be used in DRBM
to training\cite{taylor1}. Gradient can be compute as:
\begin{multline}
\frac{\partial log p\left ( y_{i}|x_{i} \right )}{\partial\theta } =\sum_{j}sigm\left ( o_{yj}\left ( x_{i} \right ) \right )\frac{\partial o_{yj}\left ( x_{i} \right )}{\partial\theta}
\\ -\sum_{i,y^{*}}sigm\left ( o_{y^{*}}\left ( x_{i} \right ) \right )p\left ( y^{*}|x_{i} \right )\frac{\partial o_{yj}\left ( x_{i} \right )}{\partial\theta}
\end{multline}

Where $o_{yj}\left ( x \right )=c_{j}+\sum_{k}W_{jkxk} + U_{jy}$. Stochastic gradient descent
optimization can be done by computed this gradient with
efficient. Fine-tuning the top RBM of DBN is using this
approach\cite{hinton1}.
\subsection{Free Energy in Persistent Contrastive Divergence}
Gibbs sampling is an appropriate method as in RBM
because each unit in RBM is independent from other units.
CD, PCD, or FEPCD\cite{keyvanrad2} can be used to obtain proper
samples from the model because impossible to use Gibbs
sampling.
In PCD methods, many persistent chain can be run in
parallel. If we can define the criteria for good chain, and for
computing the gradient sample would be more accurate.
Defining the best chain by using the free energy is as
follows \cite{hinton2} :
\begin{multline}
F\left ( v \right )=-\sum_{i}v_{i}a_{i}-\sum_{j}q_{j}I_{j}+
\\ \sum_{j}\left ( q_{j}log q_{j}+\left ( 1-q_{j} \right )log\left ( 1-q_{j} \right ) \right )
\end{multline}

Sum of input to hidden unit $j$ is equal with $I_{j}=b_{j}+\sum v_{i}w_{ij}$
and $q_{j}=q(I_{j})$ is equal to activation probability of hidden unit
$h_{j}$ given $v$ and $q$ is logistic function. An equivalent and
simpler equation for computing $F(v)$ is as follows:
\begin{equation}
F(v)=-\sum_{i}v_{i}a_{i}-\sum_{j}log(1+e^{I_{j}})
\end{equation}
\section{Proposed Method}
Fig. 1. Show proposed method in this paper. The proposed
method including three steps as follows
\begin{enumerate}
	\item Train the first RBM by inputting the original data and
	fixing up the parameters of this RBM. Then we use
	these output as the input of the second RBM and the
	rest can be done in the same manner.
	\item We use modified the RBM with normal sparse.
	\item Fine-tuning: using Back Propagation (BP). We use
	gradient-descent algorithm to revise the weight
	matrix of the whole network.
	
\end{enumerate}

\section{Experimental Setup}
The experimental study is carried out by used DeeBNet
Matlab toolbox\cite{keyvanrad3}. Also we used the new FEPCD method to
approximate the gradient of the log likelihood.
We use DBN architecture with 784 as input, 500 hidden
unit, 500 hidden unit, 2000 hidden unit, and 10 as output. This
architecture we used in each experiment.
\subsection{Data}
To evaluate the different DBN models on the the problem
of classifying images of digits, we used MNIST dataset.
MNIST dataset includes images of handwritten digits\cite{lecun} (10
classes of digits 0-9). Each digit was cared to be located in the
center of each 28*28 image. The image pixels have discrete
values between 0 and 255 hat most of them have the values at
the edge of this interval. The dataset was divided to train and
test part including 60,000 and 10,000 images respectively. Inour experiment, these discrete values have been mapped to
interval [0-1] using min-max normalization method. For
preliminary and intermediate we use small MNIST. Small
MNIST is part of MNIST dataset but only 6000 training data
and 1000 testing data.
\section{Experiment Result And Analysis}
In this section we describe the performance of DBN by
comparing with other modified DBN. We took 3 step in this
experiment, preliminary, intermediate and final result. Every
analysis of each step be a background to the next step. We use
normal sparse for sparse generative and sparse discriminative.
\subsection{Preliminary Result}
In this step, we combine generative RBM, normal sparse
generative RBM, discriminative RBM, and normal sparse
generative RBM to find the best architecture to obtain best
accuracy. Table \ref{preRes} shown the accuracy obtained. Fig 2. Feature
sparsity for 2 nd Fig 3. Feature sparsity for 4 th experiment. Fig
4. Show reconstruct for 1st experiment. Fig 5. Reconstruct for
4 th experiment.

\begin{table}[ht]
		\caption{Accuracy Obtained In Preliminary Result}
	\centering

	\label{preRes}
	\begin{tabular}{|l|l|c|c|c|c|}
		\hline
		\multicolumn{1}{|c|}{\textbf{\begin{tabular}[c]{@{}c@{}}Exp.\\ No\end{tabular}}} & \multicolumn{1}{c|}{\textbf{DBN Structure}} & \textbf{\begin{tabular}[c]{@{}c@{}}Error\\ Before\\ BP\end{tabular}} & \textbf{\begin{tabular}[c]{@{}c@{}}Error\\ After\\ BP\end{tabular}} & \textbf{\begin{tabular}[c]{@{}c@{}}Accuracy\\ (\%)\end{tabular}} & \textbf{Epoch} \\ \hline
		1                                                                                & Input- SG-SG-SD                             & 0.568                                                                & 0.055                                                               & 99.945                                                           & 200            \\ \hline
		2                                                                                & Input-G-G-D                                 & 0.063                                                                & 0.044                                                               & 99.956                                                           & 175            \\ \hline
		3                                                                                & Input-SG-SG-D                               & 0.063                                                                & 0.095                                                               & 99.905                                                           & 200            \\ \hline
		4                                                                                & Input-G-G-SD                                & 0.067                                                                & 0.038                                                               & 99.962                                                           & 124            \\ \hline
		5                                                                                & Input-SG-G-SD                               & 0.31                                                                 & 0.045                                                               & 99.955                                                           & 200            \\ \hline
		6                                                                                & Input-G-SG-D                                & 0.115                                                                & 0.048                                                               & 99.952                                                           & 200            \\ \hline
	\end{tabular}
\end{table}



\subsection{Intermediate Result}
Based on preliminary result, we know that the best result is
using input-generative-generative-Sparse Discriminative with
99.962% accuracy and 124 epoch. In intermediate experiment,
we do experiment with another likelihood method, named
FEPCD. In this experiment we implement FEPCD in our DBN
structure and compare with CD. Experiment result shown in
table II. Fig 6-9. shown that the confusion matrix for each
experiment.
\begin{table}[ht]
	\centering
	\caption{Accuracy Obtained In Intermediate Result}
	\label{interRes}
	\begin{tabular}{|l|l|l|l|l|}
		\hline
		\multicolumn{1}{|c|}{\textbf{\begin{tabular}[c]{@{}c@{}}Exp.\\ No\end{tabular}}} & \multicolumn{1}{c|}{\textbf{DBN Structure}}                                  & \multicolumn{1}{c|}{\textbf{\begin{tabular}[c]{@{}c@{}}Error\\ Before\\ BP\end{tabular}}} & \multicolumn{1}{c|}{\textbf{\begin{tabular}[c]{@{}c@{}}Accuracy\\ (\%)\end{tabular}}} & \multicolumn{1}{c|}{\textbf{Epoch}} \\ \hline
		1                                                                                & Input- G(CD)-G(CD)-SD(CD)                                                    & 0.0850                                                                                    & 99.8                                                                                  & 154                                 \\ \hline
		2                                                                                & \begin{tabular}[c]{@{}l@{}}Input-G(FEPCD)-G(FEPCD)\\ -SD(FEPCD)\end{tabular} & 0.1000                                                                                    & 99.9                                                                                  & 200                                 \\ \hline
		3                                                                                & Input- G(CD)-G(CD)-SD(FEPCD)                                                 & 0.0890                                                                                    & 99.8                                                                                  & 116                                 \\ \hline
		4                                                                                & \begin{tabular}[c]{@{}l@{}}Input- G(FEPCD)-G(FEPCD)\\ -SD(CD)\end{tabular}   & 0.1090                                                                                    & 99.8                                                                                  & 125                                 \\ \hline
	\end{tabular}
\end{table}
\subsection{Final Result}
From intermediate result we got conclusion that using
FEPCD in all of RBM is better that CD, but its consuming
time. For final result we did experiment for:
\begin{itemize}
	\item Input- Generative(CD) - Generative(CD) - Sparse
	Generative(CD)
	\item Input-Generative(CD)-Generative(CD)-Sparse
	Generative(FEPCD)
\end{itemize}

We choose Input - Generative(CD) - Generative(CD) -
Sparse Generative (FEPCD) because based on intermediate
result, this structure give good accuracy and smallest epoch.
The result of experiment shown in table III. Confusion matrix
for all experiment shown in fig 10 and fig 11. For final
experiment, we use full MNIST dataset.
\begin{table}[ht]
	\centering
	\caption{Accuracy Obtained In Final Result}
	\label{finRes}
	\begin{tabular}{|l|l|l|l|l|}
		\hline
		\multicolumn{1}{|c|}{\textbf{\begin{tabular}[c]{@{}c@{}}Exp.\\ No\end{tabular}}} & \multicolumn{1}{c|}{\textbf{DBN Structure}}                                  & \multicolumn{1}{c|}{\textbf{\begin{tabular}[c]{@{}c@{}}Error\\ Before\\ BP\end{tabular}}} & \multicolumn{1}{c|}{\textbf{\begin{tabular}[c]{@{}c@{}}Accuracy\\ (\%)\end{tabular}}} & \multicolumn{1}{c|}{\textbf{Epoch}} \\ \hline
		1                                                                                & Input- G(CD)-G(CD)-SD(CD)                                                    & 0.0850                                                                                    & 99.8                                                                                  & 154                                 \\ \hline
		2                                                                                & \begin{tabular}[c]{@{}l@{}}Input-G(FEPCD)-G(FEPCD)\\ -SD(FEPCD)\end{tabular} & 0.1000                                                                                    & 99.9                                                                                  & 200                                 \\ \hline
		3                                                                                & Input- G(CD)-G(CD)-SD(FEPCD)                                                 & 0.0890                                                                                    & 99.8                                                                                  & 116                                 \\ \hline
		4                                                                                & \begin{tabular}[c]{@{}l@{}}Input- G(FEPCD)-G(FEPCD)\\ -SD(CD)\end{tabular}   & 0.1090                                                                                    & 99.8                                                                                  & 125                                 \\ \hline
	\end{tabular}
\end{table}
\section{Conclusion}
This paper investigates the best structure in combination
normal sparse and discriminative deep belief network. Our
experimental result show that DBN with input- generative
(Contrastive Divergence) - generative (Contractive
Divergence) – normal sparse discriminative (Contractive
Divergence) give the best accuracy. It is obtained 99.9862%.
Sparse Discriminative make sparse feature. Sparse feature
give better accuracy that non sparse feature.


% use section* for acknowledgment
\section*{Acknowledgment}


The authors would like to thank...

\bibliographystyle{unsrt}
\bibliography{pustaka}






% that's all folks
\end{document}


